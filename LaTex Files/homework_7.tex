\documentclass{article}
\usepackage{amsmath,amssymb,amsthm,latexsym,paralist,url}
\usepackage[margin=1in]{geometry}
\usepackage{tikz}
\usetikzlibrary{arrows,automata}
\usepackage{csquotes}

\theoremstyle{definition}
\newtheorem{problem}{Problem}
\newtheorem*{solution}{Solution}
\newtheorem*{resources}{Resources}


\newcommand{\honor}{\noindent \textbf{Aggie Honor Statement: }On my honor as an Aggie, I have neither
  given nor received any unauthorized aid on any portion of the academic work included in this assignment.
}

 
\newcommand{\checklist}{\noindent\textbf{Checklist:}
Did you...
\begin{compactenum}
\item abide by the Aggie Honor Code?
\item solve all problems?
\item start a new page for each problem?
\item show your work clearly?
\item type your solution?
\item submit a PDF to gradescope?
\end{compactenum}
}

\newcommand{\problemset}[1]{\begin{center}\textbf{Homework #1}\end{center}}
\newcommand{\duedate}[1]{\begin{quote}\textbf{Due: #1} on gradescope (\url{gradescope.com}). \\You must show your work in order to receive credit.\end{quote}}

%%% CONSTANTS
\newcommand{\mysemester}[0]{Spring 2018}
\newcommand{\mysectionnumber}[0]{501,502}
\newcommand{\myname}[0]{Hunter Cleary}
\newcommand{\homeworknumber}[0]{7}

%%% HEADERS & FOOTERS
\usepackage{fancyhdr} % This should be set AFTER setting up the page geometry
\pagestyle{fancy} % options: empty , plain , fancy
\renewcommand{\headrulewidth}{0pt} % customise the layout...
\lhead{CSCE 222-\mysectionnumber}\chead{Homework \homeworknumber}\rhead{\myname}
\lfoot{}\cfoot{\thepage}\rfoot{}

\title{CSCE 222: Discrete Structures for Computing\\Section \mysectionnumber\\\mysemester}
\author{\myname}
\date{}

\begin{document}

\maketitle
\problemset{\homeworknumber}
\duedate{11 March 2018 (Sunday) before 11:59 p.m.}
\bigskip

\honor
\bigskip

\checklist

% functions
\begin{problem} (34 points)
\begin{compactenum}
\item Let $f(x) = ax + b$ and $g(x) = cx+d$, where $a,b,c,d$ are constants.  Determine necessary and sufficient conditions on the constants $a,b,c,d$ so that $f \circ g = g \circ f$.
\item Give 2 different non-trivial examples of $f$ and $g$ that satisfy $f \circ g = g \circ f$.\\
($a=c, b=d$ is a trivial solution)
\end{compactenum}
\end{problem}

\begin{solution}\ \\
\ \\
1. $f \circ g = f(g(x)) \therefore f\circ g = a(cx+d)+b$ \ \\ 
\indent $g \circ f = g(f(x)) \therefore g \circ f = c(ax+b) + d$\ \\
\indent so: $c(ax+b) + d = a(cx+d) + b$\ \\
\indent $ad + b = cb +d$ | are the necessary and sufficient conditions for the variables.\ \\ \ \\
2. Set #1 \{a=5,b=4,c=7,d=6\} $34=34$\ \\
\indent Set #2 \{a=2,b=0,c=4,d=4 \} $5=5$
\end{solution}

\newpage

% algorithms
\begin{problem}(33 points)\\ 
The \textbf{selection sort} begins by finding the least element in the list.  This element is moved to the front.  Then the least element among the remaining elements is found and put into the second position.  This procedure is repeated until the entire list has been sorted.
\begin{compactenum}
\renewcommand{\theenumi}{\alph{enumi}}
\item Sort this list using the selection sort: $ 6,7,1,3,2,5,4$.  Show the state of the list after every step.
\item Write the selection sort algorithm in pseudocode.
\end{compactenum}
\end{problem}

\begin{solution}\ \\
(a).
\begin{compactenum}\
\item \{6,7,1,3,2,5,4\}\ \\
\item \{1,6,7,3,2,5,4\}\ \\
\item \{1,2,6,7,3,5,4\}\ \\
\item \{1,2,3,6,7,5,4\}\ \\
\item \{1,2,3,4,6,7,5\}\ \\
\item \{1,2,3,4,5,6,7\}\ \\
\item \{1,2,3,4,5,6,7\}\ \\
\end{compactenum}
(b).
\begin{compactenum}
\indent The set of numbers consists of index 0 through n.\ \\
\indent Start with a count of 0.\ \\
\item 1. Find the smallest number from current count through n.\ \\
\item 2. Place the smallest value in front of the set just compared.\ \\
\item 3. If count = n, stop. If not, increase count by one then go to step 1.\ \\
\end{compactenum}
\end{solution}

\newpage

%algorithms
\begin{problem} (33 points)\\
Use pseudocode to describe an algorithm that determines whether a function from a finite set of integers to another finite set of integers is bijective.
\end{problem}

\begin{solution}\ \\
\begin{compactenum}
\item Set the function equal to f(x).
\item The set A consists of a1,a2,a3,a4...an
\item The set B consists of b1,b2,b3,b4...bm
\item If $n > m$ then the function is not bijective.
\item If f(A) = f(B) for all values then continue. If not, then function is not bijective.
\item If f(A) = b for all values then the function is bijective, otherwise it is not
\end{compactenum}
\end{solution}




\end{document}
