\documentclass{article}
\usepackage{amsmath,amssymb,amsthm,latexsym,paralist,url}
\usepackage[margin=1in]{geometry}
\usepackage{tikz}
\usetikzlibrary{arrows,automata}
\usepackage{csquotes}

\theoremstyle{definition}
\newtheorem{problem}{Problem}
\newtheorem*{solution}{Solution}


\newcommand{\problemset}[1]{\begin{center}\textbf{Problem Set #1}\end{center}}

%%% CONSTANTS
\newcommand{\mysemester}[0]{Fall 2017}
\newcommand{\mysectionnumber}[0]{501,502}
\newcommand{\problemsetnumber}[0]{1}
\newcommand{\mydate}[0]{22 January}

%%% HEADERS & FOOTERS
\usepackage{fancyhdr} % This should be set AFTER setting up the page geometry
\pagestyle{fancy} % options: empty , plain , fancy
\renewcommand{\headrulewidth}{0pt} % customise the layout...
\lhead{CSCE 222-\mysectionnumber}
\chead{Fun Problem \problemsetnumber, \mydate}
\rhead{Name: Hunter Cleary \hspace{.5in}}
\lfoot{}\cfoot{\thepage}\rfoot{}


\begin{document}

\noindent
Each fun problem is worth 2 Extra Credit points.\\
Due: 26 January 2017 (Friday) before 11:59pm on gradescope (\url{gradescope.com}).\\

\noindent
On Bamf Street, there are 5 houses in row.  Each house is a different color and the owners each have a unique name, drink they prefer, fruit they grow, and musical instrument they play.  On his daily walks around the neighborhood, Alan Turing walks past the houses on Bamf Street and has noted the following:
\begin{compactitem}
\item Ada Lovelace lives in the first house.
\item Ada Lovelace lives next to the red house.
\item Grace Hopper lives in the white house.
\item The green house's neighbour to the left is the yellow house.
\item The owner of the green house drinks coffee.
\item The person living in the middle house drinks beer.
\item Shafi Goldwasser drinks tea.
\item The person who grows peaches plays the trumpet.
\item The person who grows avocados has a neighbour who drinks wine.
\item The person who grows avocados lives next to the one who plays the cello.
\item The person who grows berries drinks water.
\item The person who grows cherries lives in a blue house.
\item Dorothy Denning grows pears.
\item The person who plays the piano lives next to the one who grows cherries.
\item Frances Allen plays the flute.
\end{compactitem}
Just the other day, as Alan Turing was walking along Bamf Street, he heard someone playing bongos.  When he returned home, he sat down with a pencil and some paper and was able to determine which of the 5 people on Bamf Street plays the bongos.\\
\textbf{Who plays the bongos?}\footnote{The complete answer will paint the full picture: who lives where and what is their associated color, drink, fruit, and musical instrument.}
\newline
\newline

\noindent\textbf{Solution}
\newline
\noindent
The answer was generated by creating a 5 x 6 spreadsheet. Every option was listed in the proper category. Each item that pertained to a house was logically ruled out until only one remained for each house.

% Table generated by Excel2LaTeX from sheet 'Sheet1'
\begin{table}[htbp]
  \centering
  \caption{Neighborhood}
    \begin{tabular}{llllll}
    House \# & \multicolumn{1}{r}{1} & \multicolumn{1}{r}{2} & \multicolumn{1}{r}{3} & \multicolumn{1}{r}{4} & \multicolumn{1}{r}{5} \\
    Name  & Ada Lovelace &  Shafi Goldwasser & Grace Hopper & Frances Allen & Dorothy Denning \\
    Color & Blue  & Red   & White & Yellow & Green \\
    Fruit & Cherries & Avacadoes & Peaches & Berries & Pears \\
    Drink & Wine  & Tea   & Beer  & Water & Coffee \\
    Instrument & Cello & Piano & Trumpet & Flute & Bongos \\
    \end{tabular}%
  \label{tab:addlabel}%
\end{table}%

\newline
\newline

\noindent
Dorothy Denning plays the bongos in house #5.


\end{document}
