\documentclass{article}
\usepackage{amsmath,amssymb,amsthm,latexsym,paralist,url}
\usepackage[margin=1in]{geometry}
\usepackage{tikz}
\usetikzlibrary{arrows,automata}
\usepackage{csquotes}

% settings for algorithm2e psuedocode style
\usepackage[ruled, linesnumbered, commentsnumbered, noend]{algorithm2e}
% a few parameters that are set to typeset pseudocode
\SetAlCapNameSty{textsc}
\SetProcNameSty{textsc}
\SetProcArgSty{textsc}
\SetKw{KwBy}{by}

\theoremstyle{definition}
\newtheorem{problem}{Problem}
\newtheorem*{solution}{Solution}
\newtheorem*{resources}{Resources}


\newcommand{\honor}{\noindent \textbf{Aggie Honor Statement: }On my honor as an Aggie, I have neither
  given nor received any unauthorized aid on any portion of the academic work included in this assignment.
}

 
\newcommand{\checklist}{\noindent\textbf{Checklist:}
Did you...
\begin{compactenum}
\item abide by the Aggie Honor Code?
\item solve all problems?
\item start a new page for each problem?
\item show your work clearly?
\item type your solution?
\item submit a PDF to gradescope and correctly assign problems to pages?
\end{compactenum}
}

\newcommand{\problemset}[1]{\begin{center}\textbf{Homework #1}\end{center}}
\newcommand{\duedate}[1]{\begin{quote}\textbf{Due: #1} on gradescope (\url{gradescope.com}). \\You must show your work in order to receive credit.\end{quote}}

%%% CONSTANTS
\newcommand{\mysemester}[0]{Spring 2018}
\newcommand{\mysectionnumber}[0]{501,502}
\newcommand{\myname}[0]{Hunter Cleary}
\newcommand{\homeworknumber}[0]{13}

%%% HEADERS & FOOTERS
\usepackage{fancyhdr} % This should be set AFTER setting up the page geometry
\pagestyle{fancy} % options: empty , plain , fancy
\renewcommand{\headrulewidth}{0pt} % customise the layout...
\lhead{CSCE 222-\mysectionnumber}\chead{Homework \homeworknumber}\rhead{\myname}
\lfoot{}\cfoot{\thepage}\rfoot{}

\title{CSCE 222: Discrete Structures for Computing\\Section \mysectionnumber\\\mysemester}
\author{\myname}
\date{}

\begin{document}

\maketitle
\problemset{\homeworknumber}
\duedate{4 May 2018 (Friday) before 11:59 p.m.}
\bigskip

\honor
\bigskip

\checklist
\bigskip

%\newpage

% Recursive Algorithms
\begin{problem} (25 points)
\begin{compactenum}
\renewcommand{\theenumi}{\alph{enumi}}
\item Give a recursive algorithm for finding the sum of the first $n$ odd positive integers.\\
\textit{Do not use the closed form solution, do not use a loop.}
\item Prove that your algorithm is correct.
\end{compactenum}
\end{problem}

\begin{solution}\ \\
 function sum()\ \\
 parameters: n\ \\
 \ \\
 \noindent
 if n = 1\ \\
 \indent return 1\ \\
 \ \\
 else \ \\
 \indent return $2(n-1) + 1 +$ sum$(n-1)$\ \\
 \ \\
 \ \\
 sum(5) = 9 + 7 + 5 + 3 + 1 = 25\ \\
 sum(4) = 7 + 5 + 3 + 1 = 16\ \\
 sum(3) = 5 + 3 + 1 = 9\ \\
 sum(2) = 3 + 1 = 4 \ \\
 sum(1) = 1\ \\
\end{solution}

\newpage

% Recursive Algorithms
\begin{problem} (25 points)\\
Trace the execution of the following recursive algorithm on input $b=2,n=10,m=7$.  That is, show all the steps it uses to compute $2^{10} \pmod{7}$.
\begin{procedure}
\caption{mpower($b,n,m$ : integers with $b>0,n\geq 0,m\geq 2$)}
\DontPrintSemicolon
\If{$n=0$}{
\Return{$1$}
}
\ElseIf{n is even}{
\Return{$mpower(b,n/2,m)^2 \mod m$}
}\Else{
\Return{$((mpower(b,(n-1)/2,m)^2 \mod m) \cdot (b \mod m)) \mod m$}
}
\tcp*[l]{output is $b^n \mod m$}
\end{procedure}
\end{problem}

\begin{solution}\ \\
\ \\
$MPOWER(2,10,7)$\ \\
n is even\ \\
$MPOWER(2,5,7)^2$ $\% 7$\ \\
n is not even\ \\
$MPOWER(2,5/2,7)^2$ $\% 7$(2 \% 7) \% 7)\ \\
$MPOWER(2,2,7)^2$ $\% 7$(2 \% 7) \% 7)\ \\
n is even\ \\
$MPOWER(2,2/2,7)^2$ $\% 7$)\ \\
$MPOWER(2,1,7)^2$ $\% 7$)\ \\
n is not even\ \\
$MPOWER(2,1/2,7)^2$ $\% 7$(2 \% 7) \% 7)\ \\
$MPOWER(2,0,7)^2$ $\% 7$(2 \% 7) \% 7)\ \\
$MPOWER(2,0,7) = 1$
\ \\
\ \\
$2^{10} \ \% \ 7 =2$
\ \\ 
\end{solution}

\newpage

% Counting
\begin{problem} (25 points)\\
How many functions are there from the set $\{0,1\}^n$ to the set $\{0,1\}$?  That is, how many boolean functions of $n$ variables are there that map to a single boolean value?  \textit{Example: $\neg$ is a boolean function of 1 variable, $\wedge, \vee, \to, \oplus, \leftrightarrow$ are boolean functions of 2 variables. Hint: every function has different truth table.}
\end{problem}

\begin{solution}\ \\
A boolean of one variable has 4 functions that can be constructed. (TRUE / FALSE). A boolean of two variables has 16 functions that can be constructed.\ \\
$
\ \\
p \ q\|F1 \ F2 \ F3 \ F4 \  F5 \ F6 \ F7 \ F8\ \\
0 \  0\|  0 \  0 \  0  \ 0 \  0 \  0 \  0 \  0\ \\
0 \  1\|  0 \  0 \  0  \ 0  \ 1 \  1  \ 1  \ 1\ \\
1 \  0\|  0 \  0 \ 1 \  1 \  0 \  0 \  1  \ 1\ \\
1 \  1\|  0  \ 1 \  0  \ 1\   0 \  1 \  0  \ 1\ \\
\ \\
$
All functions map to the single boolean value of true or false.
\end{solution}

\newpage

% Counting
\begin{problem} (25 points)\\
Prove that at a party where there are at least two people, there are two people who know the same number of other people there.
\end{problem}

\begin{solution}\ \\ 
\ \\
There will be n people at the party. The max number of people a person can know is n-1. The fewest they can know is 1. (How they get invited.) The are n-1 numbers of people that can be known and n people, so the same number must be applied to two different individiuals.\ \\ 
\ \\
-$>$ Pidgeonhole Principle $<$-
\end{solution}


\end{document}
