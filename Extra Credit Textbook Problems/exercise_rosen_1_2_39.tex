\documentclass{article}
\usepackage{amsmath,amssymb,amsthm,latexsym,paralist,url}
\usepackage[margin=1in]{geometry}

\theoremstyle{definition}
\newtheorem*{problem}{Problem}
\newtheorem*{solution}{Solution}

\begin{document}

% Hunter Cleary
% Chapter 1.2, Problem 39
\begin{problem}[Rosen.1.2.39]\ \\
\ \\
Freedonia has fifty senators. Each senator is either honest
or corrupt. Suppose you know that at least one of the Freedonian
senators is honest and that, given any two Freedonian
senators, at least one is corrupt. Based on these
facts, can you determine how many Freedonian senators
are honest and how many are corrupt? If so, what is the
answer?\ \\
\begin{compactenum}
\renewcommand{\theenumi}{\alph{enumi}}
\end{compactenum}
\end{problem}

\begin{solution}\ \\
\ \\
\textit{Propositions}:\ \\
P: Each senator is honest\ \\
Q: Each senator is corrupt\ \\
\ \\
If one of the Freedonian senators is honest and with two Freedonian senators, at least one is corrupt.\ \\
\ \\
One senator is honest and the other 49 senators are corrupt. Since any 2 random senators contains at least one corrupt, there can't be a chance to pull 2 honest senators.\ \\
\ \\
\begin{compactenum}
\renewcommand{\theenumi}{\alph{enumi}}  





\end{compactenum}
\end{solution}

\end{document}