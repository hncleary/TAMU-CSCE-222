\documentclass{article}
\usepackage{amsmath,amssymb,amsthm,latexsym,paralist,url}
\usepackage[margin=1in]{geometry}

\theoremstyle{definition}
\newtheorem*{problem}{Problem}
\newtheorem*{solution}{Solution}

\begin{document}

% Hunter Cleary
% Chapter 1.2, Problem 35
\begin{problem}[Rosen.1.2.35]\ \\
\ \\
A detective has interviewed four witnesses to a crime.
From the stories of the witnesses the detective has concluded
that if the butler is telling the truth then so is the
cook; the cook and the gardener cannot both be telling the
truth; the gardener and the handyman are not both lying;
and if the handyman is telling the truth then the cook is
lying. For each of the four witnesses, can the detective determine
whether that person is telling the truth or lying?\ \\
Explain your reasoning.\ \\
\begin{compactenum}
\renewcommand{\theenumi}{\alph{enumi}}
\end{compactenum}
\end{problem}

\begin{solution}\ \\
\ \\
\begin{compactenum}
\renewcommand{\theenumi}{\alph{enumi}} 
\textit{Propositions}:\ \\
B: truth spoken by butler\ \\
C: truth spoken by cook\ \\
G: truth spoken by gardener\ \\
H: truth spoken by handyman\ \\
\ \\
\textit{Expressions}:\ \\
$B \rightarrow C $\ \\
$\neg(C \wedge G) $\ \\
$G \vee H$\ \\
$H \rightarrow \neg C$\ \\
\ \\
Suppose that B is truthful. Then C is also true. This implies that G is false. Then H will be true. Then the fourth proposition is not valid. Therefore, B and C cannot be true. \textbf{The butler and cook are lying}.\ \\
\ \\
Therefore, \textbf{either the handyman or the gardener is telling the truth} ( $G \vee H$)
\ \\




\end{compactenum}
\end{solution}

\end{document}