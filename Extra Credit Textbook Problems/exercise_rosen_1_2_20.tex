\documentclass{article}
\usepackage{amsmath,amssymb,amsthm,latexsym,paralist,url}
\usepackage[margin=1in]{geometry}

\theoremstyle{definition}
\newtheorem*{problem}{Problem}
\newtheorem*{solution}{Solution}

\begin{document}

% Hunter Cleary
% Chapter 1.2, Problem 20
\begin{problem}[Rosen.1.2.20]\ \\
% Delete from 19 - 23
\ \\
Exercises 19 $\rightarrow$ 23 relate to inhabitants of the island of knights
and knaves created by Smullyan, where knights always tell
the truth and knaves always lie. You encounter two people,
A and B. Determine, if possible, what A and B are if they
address you in the ways described. If you cannot determine
what these two people are, can you draw any conclusions?\ \\
\ \\
\textbf{20.} A says “The two of us are both knights” and B says “A is a knave.”\ \\
\begin{compactenum}
\renewcommand{\theenumi}{\alph{enumi}}

\end{compactenum}
\end{problem}

\begin{solution}\ \\
\ \\
\textit{Propositions}:\ \\
p: A is a knight\ \\
q: B is a knight\ \\
\ \\
$\neg p$: A is a knave\ \\
$\neg q$: B is a knave\ \\
\ \\
Consider the possibility that A is a knight. The statement $p \wedge q$ is True. Therefore, B is also a knight. But if B is truthful then $\neg p$ is true, which is contradictory.\ \\
\ \\
Consider the possibility of A being a knave. The statement of B is then true and the system is consistent. \textbf{B is a knight and A is a knave}.
\begin{compactenum}
\renewcommand{\theenumi}{\alph{enumi}}


\end{compactenum}
\end{solution}

\end{document}