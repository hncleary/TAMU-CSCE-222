\documentclass{article}
\usepackage{amsmath,amssymb,amsthm,latexsym,paralist,url}
\usepackage[margin=1in]{geometry}

\theoremstyle{definition}
\newtheorem*{problem}{Problem}
\newtheorem*{solution}{Solution}

\begin{document}

% Hunter Cleary
% Chapter 1.2, Problem 28
\begin{problem}[Rosen.1.2.28]\ \\
% Delete from 19 - 23
\ \\
Exercises 24 $\rightarrow$ 31 relate to inhabitants of an island on which
there are three kinds of people: knights who always tell the
truth, knaves who always lie, and spies (called normals by
Smullyan [Sm78]) who can either lie or tell the truth. You
encounter three people, A, B, and C. You know one of these
people is a knight, one is a knave, and one is a spy. Each of the
three people knows the type of person each of other two is. For
each of these situations, if possible, determine whether there
is a unique solution and determine who the knave, knight, and
spy are. When there is no unique solution, list all possible
solutions or state that there are no solutions.\ \\
\ \\
\textbf{28.} A says “I am the knight,” B says, “A is not the knave,”
and C says “B is not the knave.”
\begin{compactenum}
\renewcommand{\theenumi}{\alph{enumi}}

\end{compactenum}
\end{problem}

\begin{solution}\ \\
\ \\
Consider a situation that A is the knight. If A says that “I am the knight” it is a true statement. If B says that “A is not the knave” it is also a true statement. So assume that B is a spy who tells truth. If C says that “B is not the knave” then this statement is also true. This is impossible because C would need to be a knight or a spy.\ \\
\ \\
Consider a situation that A is a knave. If A says that “I am the knight” then this is a false statement. If B says that “A is not the knave” then this is false statement. B is a spy who lies. If C says that “B is not the knave” then this statement is true. So C is the knight.\ \\
\ \\
\textbf{A is the knave, B is the spy and C is the knight}.





\begin{compactenum}
\renewcommand{\theenumi}{\alph{enumi}}


\end{compactenum}
\end{solution}

\end{document}