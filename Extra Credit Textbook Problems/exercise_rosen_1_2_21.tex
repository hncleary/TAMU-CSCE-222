\documentclass{article}
\usepackage{amsmath,amssymb,amsthm,latexsym,paralist,url}
\usepackage[margin=1in]{geometry}

\theoremstyle{definition}
\newtheorem*{problem}{Problem}
\newtheorem*{solution}{Solution}

\begin{document}

% Hunter Cleary
% Chapter 1.2, Problem 21
\begin{problem}[Rosen.1.2.21]\ \\
% Delete from 19 - 23
\ \\
Exercises 19 $\rightarrow$ 23 relate to inhabitants of the island of knights
and knaves created by Smullyan, where knights always tell
the truth and knaves always lie. You encounter two people,
A and B. Determine, if possible, what A and B are if they
address you in the ways described. If you cannot determine
what these two people are, can you draw any conclusions?\ \\
\ \\
\textbf{21.} A says “I am a knave or B is a knight” and B says nothing
\begin{compactenum}
\renewcommand{\theenumi}{\alph{enumi}}

\end{compactenum}
\end{problem}

\begin{solution}\ \\
\ \\
\textit{Propositions}:\ \\
p: A is a knight\ \\
q: B is a knight\ \\
\ \\
$\neg p$: A is a knave\ \\
$\neg q$: B is a knave\ \\
\ \\
Consider the possibility that A is knight. He must tell the truth and one of his statements is true. Since A is a knight then his statement 'I am a knave' is false but 'B is a knight' is true. Therefore, A is a knight and B is knight.\ \\
\ \\
Consider the possibility that A is a knave. A must lie and both statements would be false. When A tells that “I am a knave”, it is true. Therefore, A is not a knave.\ \\
\ \\
Therefore, \textbf{A is a knight and B is knight}.

\begin{compactenum}
\renewcommand{\theenumi}{\alph{enumi}}


\end{compactenum}
\end{solution}

\end{document}