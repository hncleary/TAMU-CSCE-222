\documentclass{article}
\usepackage{amsmath,amssymb,amsthm,latexsym,paralist,url}
\usepackage[margin=1in]{geometry}

\theoremstyle{definition}
\newtheorem*{problem}{Problem}
\newtheorem*{solution}{Solution}

\begin{document}

% Hunter Cleary
% Chapter 1.2, Problem 26
\begin{problem}[Rosen.1.2.26]\ \\
% Delete from 19 - 23
\ \\
Exercises 24 $\rightarrow$ 31 relate to inhabitants of an island on which
there are three kinds of people: knights who always tell the
truth, knaves who always lie, and spies (called normals by
Smullyan [Sm78]) who can either lie or tell the truth. You
encounter three people, A, B, and C. You know one of these
people is a knight, one is a knave, and one is a spy. Each of the
three people knows the type of person each of other two is. For
each of these situations, if possible, determine whether there
is a unique solution and determine who the knave, knight, and
spy are. When there is no unique solution, list all possible
solutions or state that there are no solutions.\ \\
\ \\
\textbf{26.} A says “I am the knave,” B says “I am the knave,” and C
says “I am the knave.”
\begin{compactenum}
\renewcommand{\theenumi}{\alph{enumi}}

\end{compactenum}
\end{problem}

\begin{solution}\ \\
\ \\
%\textit{Propositions}:\ \\
\noindent A, B, and C claim that they are knaves. None can be knights. None can be knaves. There is \textbf{no solution} to this problem. One must be knight, but none are truthful. must be a knave, but knaves lie so the scenario cannot exist.





\begin{compactenum}
\renewcommand{\theenumi}{\alph{enumi}}


\end{compactenum}
\end{solution}

\end{document}