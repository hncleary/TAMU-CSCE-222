\documentclass{article}
\usepackage{amsmath,amssymb,amsthm,latexsym,paralist,url}
\usepackage[margin=1in]{geometry}

\theoremstyle{definition}
\newtheorem*{problem}{Problem}
\newtheorem*{solution}{Solution}

\begin{document}

% Hunter Cleary
% Chapter 1.2, Problem 33
\begin{problem}[Rosen.1.2.33]\ \\
\ \\
Steve would like to determine the relative salaries of three
coworkers using two facts. First, he knows that if Fred
is not the highest paid of the three, then Janice is. Second,
he knows that if Janice is not the lowest paid, then
Maggie is paid the most. Is it possible to determine the
relative salaries of Fred, Maggie, and Janice from what
Steve knows? If so, who is paid the most and who the
least? Explain your reasoning\ \\
\begin{compactenum}
\renewcommand{\theenumi}{\alph{enumi}}
\end{compactenum}
\end{problem}

\begin{solution}\ \\
\ \\
\begin{compactenum}
\renewcommand{\theenumi}{\alph{enumi}} 

Steve: Fred is not paid the highest. Janice will be the one which means she is not the lowest, so Maggie is highest paid. This is a contradiction. It can be said that Fred is highest paid.\ \\
\ \\
Next, Janice will not be the lowest paid. It is implied that Maggie will be highest paid. This is a contradiction. Thus, Janice is lowest paid.\ \\
\ \\
\textbf{Fred is highest paid, Maggie will be next and then least will be Janice}.\ \\
\ \\


\end{compactenum}
\end{solution}

\end{document}