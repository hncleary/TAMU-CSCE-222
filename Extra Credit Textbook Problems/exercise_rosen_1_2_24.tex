\documentclass{article}
\usepackage{amsmath,amssymb,amsthm,latexsym,paralist,url}
\usepackage[margin=1in]{geometry}

\theoremstyle{definition}
\newtheorem*{problem}{Problem}
\newtheorem*{solution}{Solution}

\begin{document}

% Hunter Cleary
% Chapter 1.2, Problem 24
\begin{problem}[Rosen.1.2.24]\ \\
% Delete from 19 - 23
\ \\
Exercises 24 $\rightarrow$ 31 relate to inhabitants of an island on which
there are three kinds of people: knights who always tell the
truth, knaves who always lie, and spies (called normals by
Smullyan [Sm78]) who can either lie or tell the truth. You
encounter three people, A, B, and C. You know one of these
people is a knight, one is a knave, and one is a spy. Each of the
three people knows the type of person each of other two is. For
each of these situations, if possible, determine whether there
is a unique solution and determine who the knave, knight, and
spy are. When there is no unique solution, list all possible
solutions or state that there are no solutions.\ \\
\ \\
\textbf{24.} A says “C is the knave,” B says, “A is the knight,” and C
says “I am the spy.”
\begin{compactenum}
\renewcommand{\theenumi}{\alph{enumi}}

\end{compactenum}
\end{problem}

\begin{solution}\ \\
\ \\
%\textit{Propositions}:\ \\
\ \\
\noindent
First, B cannot be a knight. Suppose B becomes a knight , then A will also have to be one. But both of them cannot be knights.\ \\
\ \\
C cannot be a knight; his statement will not be true. \textbf{A has to be a knight}, and as a result of his statement,\textbf{C will be a knave}. By method of elimination, \textbf{B will be a spy}. 



\begin{compactenum}
\renewcommand{\theenumi}{\alph{enumi}}


\end{compactenum}
\end{solution}

\end{document}