\documentclass{article}
\usepackage{amsmath,amssymb,amsthm,latexsym,paralist,url}
\usepackage[margin=1in]{geometry}

\theoremstyle{definition}
\newtheorem*{problem}{Problem}
\newtheorem*{solution}{Solution}

\begin{document}

% Hunter Cleary
% Chapter 1.2, Problem 17
\begin{problem}[Rosen.1.2.17]\ \\
When three professors are seated in a restaurant, the hostess
asks them: “Does everyone want coffee?” The first
professor says: “I do not know.” The second professor
then says: “I do not know.” Finally, the third professor
says: “No, not everyone wants coffee.” The hostess comes
back and gives coffee to the professors who want it. How
did she figure out who wanted coffee?
\begin{compactenum}
\renewcommand{\theenumi}{\alph{enumi}}

\end{compactenum}
\end{problem}

\begin{solution}\ \\
\ \\
The first professor does not know the choices of the other two professors, so he says 
"I don't know". I he did not want coffee, he would know that not everyone wanted some. The same goes for the second professor. The third knows that he does not want any, so he says no to the question. \textbf{The first and second then get coffee.}
\begin{compactenum}
\renewcommand{\theenumi}{\alph{enumi}}


\end{compactenum}
\end{solution}

\end{document}