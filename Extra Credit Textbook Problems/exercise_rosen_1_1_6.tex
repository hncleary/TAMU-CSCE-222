\documentclass{article}
\usepackage{amsmath,amssymb,amsthm,latexsym,paralist,url}
\usepackage[margin=1in]{geometry}

\theoremstyle{definition}
\newtheorem*{problem}{Problem}
\newtheorem*{solution}{Solution}

\begin{document}


% Chapter 1.1, Problem 1
\begin{problem}[Rosen.1.1.6]\ \\
Suppose that Smartphone A has 256 MB RAM and 32 GB ROM and the resolution of its camera is 8MP; Smartphone B has 288 MB RAM and 64 GB ROM, and the resolution of its camera is 4 MP; and Smartphone C has 128 MB RAM and 32 GB ROM, and the resolution of its camera is 5 MP. Determine the truth value of each of these propositions.
\ \\
\begin{compactenum}
\renewcommand{\theenumi}{\alph{enumi}}

\item Smartphone B has the most RAM of these three smartphones.
\item Smartphone C had more ROM or a higher resolution camera than Smartphone B.
\item Smartphone B has more RAM, more ROM, and a higher resolution camera than smartphone A.
\item If Smartphone B has more RAM and more ROM than Smartphone C, then it also has a higher resolution camera.
\item Smartphone A has more RAM than Smartphone  if and only if Smartphone B has more RAM than Smartphone A.
\end{compactenum}
\end{problem}

\begin{solution}\ \\
\ \\
\noindent 
\textit{Propositions}:\ \\
p: Smartphone A has 256 MB RAM and 32 GB ROM and the resolution of its camera is 8MP\ \\
q: Smartphone B has 288 MB RAM and 64 GB ROM, and the resolution of its camera is 4 MP\ \\
r: Smartphone C has 128 MB RAM and 32 GB ROM, and the resolution of its camera is 5 MP\ \\
\begin{compactenum}
\renewcommand{\theenumi}{\alph{enumi}}
\item True, B has the most RAM of the 3 devices.
\item Propositions $p \vee q$ where p is false and q is true. $F \vee T \equiv T$. $\therefore$ True
\item Propositions $p \wedge q$ where p is true and q is false. $T \wedge F \equiv F$. $\therefore$ False
\item Propositions $p \xrightarrow{} q$ where p is true and q is false. $T \xrightarrow{} F \equiv F$. $\therefore$ False
\item Propositions $p \leftrightarrow q$ where p is false and q is true. $F \leftrightarrow T \equiv F$. $\therefore$ False
\end{compactenum}
\end{solution}

\end{document}