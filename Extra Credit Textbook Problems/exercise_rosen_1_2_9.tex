\documentclass{article}
\usepackage{amsmath,amssymb,amsthm,latexsym,paralist,url}
\usepackage[margin=1in]{geometry}

\theoremstyle{definition}
\newtheorem*{problem}{Problem}
\newtheorem*{solution}{Solution}

\begin{document}


% Chapter 1.1, Problem 1
\begin{problem}[Rosen.1.2.9]\ \\
Are these system specifications consistent? “The system
is in multiuser state if and only if it is operating normally.
If the system is operating normally, the kernel is functioning.
The kernel is not functioning or the system is
in interrupt mode. If the system is not in multiuser state,
then it is in interrupt mode. The system is not in interrupt
mode.”
\begin{compactenum}
\renewcommand{\theenumi}{\alph{enumi}}

\end{compactenum}
\end{problem}

\begin{solution}\ \\
\ \\
\textit{Propositions}:\ \\
p: The system is in a multi user state\ \\
q: The system is operating normally\ \\
r: The kernel is functioning\ \\
s: The system is in intrerrupt mode\ \\
\ \\
The following can be written with the given propositions:\ \\
1. $p \leftrightarrow q$\ \\
2. $q \rightarrow r $\ \\
3. $\neg r \wedge s$\ \\
4. $\neg p \rightarrow s$\ \\
5. $\neg s$ \ \\
\ \\
All will result in true except for the 4th proposition. The value of $\neg p$ is true and the value of s is false so the result is false.
\begin{compactenum}
\renewcommand{\theenumi}{\alph{enumi}}

\end{compactenum}
\end{solution}

\end{document}