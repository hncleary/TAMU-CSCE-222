\documentclass{article}
\usepackage{amsmath,amssymb,amsthm,latexsym,paralist,url}
\usepackage[margin=1in]{geometry}

\theoremstyle{definition}
\newtheorem*{problem}{Problem}
\newtheorem*{solution}{Solution}

\begin{document}

% Hunter Cleary
% Chapter 1.2, Problem 30
\begin{problem}[Rosen.1.2.30]\ \\
% Delete from 19 - 23
\ \\
Exercises 24 $\rightarrow$ 31 relate to inhabitants of an island on which
there are three kinds of people: knights who always tell the
truth, knaves who always lie, and spies (called normals by
Smullyan [Sm78]) who can either lie or tell the truth. You
encounter three people, A, B, and C. You know one of these
people is a knight, one is a knave, and one is a spy. Each of the
three people knows the type of person each of other two is. For
each of these situations, if possible, determine whether there
is a unique solution and determine who the knave, knight, and
spy are. When there is no unique solution, list all possible
solutions or state that there are no solutions.\ \\
\ \\
\textbf{30.} A says “I am not the spy,” B says “I am not the spy,” and C says “A is the spy.”
\begin{compactenum}
\renewcommand{\theenumi}{\alph{enumi}}

\end{compactenum}
\end{problem}

\begin{solution}\ \\
\ \\

\begin{compactenum}
\renewcommand{\theenumi}{\alph{enumi}}
\begin{table}[htbp]
  \centering
  \caption{Truth Table}
    \begin{tabular}{llllr}
    A     & B     & C     & Truth & \multicolumn{1}{l}{Possible?} \\
    knight & knave & spy   & T,T,F &  \\
    knight & spy   & knave & T,F,F & \multicolumn{1}{l}{Yes} \\
    knave & knight & spy   & T,T,F &  \\
    knave & spy   & knight & T,F,F &  \\
    spy   & knight & knave & F,T,T &  \\
    spy   & knave & knight & F,T,T &  \\
    \end{tabular}%
  \label{tab:addlabel}%
\end{table}%
\ \\
\textbf{A is the knight, B is the spy, and C is the knave.}

\end{compactenum}
\end{solution}

\end{document}